\documentclass[12pt,a4paper]{article}
\usepackage[utf8]{inputenc}
\usepackage[serbian]{babel}
\usepackage{geometry}
\usepackage{booktabs}
\usepackage{array}
\usepackage{tabularx}

\geometry{
    a4paper,
    left=2.5cm,
    right=2.5cm,
    top=2.5cm,
    bottom=2.5cm
}

\title{Izvještaj o izvedenoj nastavi\\Klub programiranja\\Prva Bošnjačka Gimnazija}
\author{Predavač: Bakir Činjarević}
\date{}

\begin{document}

\maketitle

\vspace{1cm}

\begin{center}
\textbf{Plan i program izvedene nastave}
\end{center}

\vspace{0.5cm}

\begin{center}
\begin{table}[h]
\centering
\begin{tabularx}{\textwidth}{|p{1cm}|X|p{2cm}|X|}
\hline
\textbf{R.br.} & \textbf{Datum} & \textbf{Broj sati} & \textbf{Tematika} \\
\hline
1 & 27. septembar 2025. & 3 & Uvod u programiranje, 2kk, binarni brojevi \\
\hline
2 & 4. oktobar 2025. & 3 & Nizovi, kompajliranje, stringovi \\
\hline
3 & 11. oktobar 2025. & 3 & Kompleksnosti i vježbanje nizova \\
\hline
4 & 18. oktobar 2025. & 3 & Kontejnerski tipovi \\
\hline
5 & 25. oktobar 2025. & 3 & Vježbanje, povezane liste \\
\hline
6 & 1. novembar 2025. & 3 & Rekurzija i kumulativne sume \\
\hline
7 & 8. novembar 2025. & 2 & Binarna pretraga i kombinatorika \\
\hline
\multicolumn{3}{|c|}{\textbf{Ukupno:}} & \textbf{20} \\
\hline
\end{tabularx}
\end{table}
\end{center}

\vspace{1cm}

\begin{flushright}
Sarajevo, \today
\end{flushright}

\end{document}

